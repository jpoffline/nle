
\tikzstyle{block} = [fill=red!20, draw,rectangle,  text centered, rounded corners, minimum height=2em]
\tikzstyle{block2} = [fill=blue!20,draw,rectangle,  text centered, rounded corners, minimum height=2.5em]
\tikzstyle{block3} = [fill=green!20,draw,rectangle,  text centered, rounded corners, minimum height=2em]
\tikzstyle{block4} = [fill=yellow!5,draw,rectangle,   dotted, text centered, rounded corners, minimum height=2em]

\tikzstyle{line} = [draw,  -latex,   thick]
\tikzstyle{line2} = [draw,  latex-,   thick]
\tikzstyle{cloud} = [ minimum height=2em]

\begin{figure}[!t]
\begin{centering}
\begin{tikzpicture}[node distance = 2cm, auto]
    % Place nodes
    \node [block2] (materialmodels) {{\bf Material models}};
    \node [block3, right of=materialmodels, node distance=5cm] (imperfect) {Viscous};
%    \node [block3, left of=materialmodels, node distance=7cm] (kv) {Visco-elastic solid};
    \node [block3, below right = 1.5cm of materialmodels, node distance=2cm] (solid) {Solid};		
    \node [block3, above left = 1.5cm of materialmodels, node distance=2.5cm] (mixtures) {Mixture};
    \node [block3, above right = 1.5cm of materialmodels, node distance=5cm] (plastic) {Plastic};    
    \node [block3, below left = 1.5cm of materialmodels, node distance=5cm] (fluids) {Fluid};
    \node [block4, below left = 0.5cm of solid, node distance=5cm] (cosmology) {\small Cosmology};
    \node [block4, below right = 0.5cm of solid, node distance=4cm] (nonlinear) {\small Compact objects};
        \node [block, above right = 1.25cm of mixtures, node distance=2cm] (solidscalar) {\small Solid+scalar};
        \node [block, right of=solidscalar, node distance=4cm] (fluidscalar) {\small Fluid+scalar};        
    % Draw edges
    \draw [line] (materialmodels) -- node{\scriptsize perfect}(solid);
    \draw [line] (materialmodels) -- node{\scriptsize example}(fluids);
    \draw [line] (materialmodels) -- node{\scriptsize imperfect}(imperfect);
    \draw [line] (materialmodels) -- node{\scriptsize extra}(mixtures);    
    \draw [line] (materialmodels) -- node{\scriptsize extra}(plastic);        
    \draw [line] (solid) -- node{\scriptsize zero rigidity}(fluids);     
    \draw [line] (solid) -- node{\scriptsize linear}(cosmology);        
    \draw [line] (solid) -- node{\scriptsize non-linear}(nonlinear);           
%    \draw [line] (kv) --  node{\scriptsize Kelvin-Voigt}(cosmology);             
    \draw [line] (mixtures) -- node{\scriptsize hyper-elastic}(solidscalar);             
    \draw [line] (solidscalar) -- node{\scriptsize sub-case}(fluidscalar);         
%    \draw [line] (kv) --  (cosmology);                                
\end{tikzpicture}
\caption{Illustration containing some of the simplest material models. This picture coarsely shows how some of the common classes of materials are related. For example, we see that a fluid is a perfect solid with zero rigidity. We have also shown that the linear theory of solids has been applied to cosmology, and the non-linear theory to compact objects (such as neutron stars). There are also imperfect materials, such as viscous solids and plastics. In addition, there are models which dynamically mix the degrees of freedom of a solid with those of a scalar; they can be categorised as ``hyper-elastic'', in the sense of Carter. There are other mixing models, such as those which mix multiple materials, but they are not considered here.}\label{fig:shem_roadmap}
\end{centering}
\end{figure}
 