\section{Application to general isotropic configurations of elastic solids}
The aim of this section is to understand the physics of elastic solids in isotropic configurations. We will study (a) elastic stars, (b) stars immersed in an elastic solid. The former problem has been studied before, but under the guise of ``anisotropic stars'', without mention of the anisotropy coming from elasticity. The latter problem is relevant for understanding how an elastic ``bath'' could affect the properties of compact objects (specifically, we will be looking out for some analogue of a ``screen''). 

See \cite{Frauendiener:2007yx}. There are some axially symmetric solutions in \cite{GRG_giulio_1993, Brito:2014hra, 1742-6596-314-1-012028, brito_thesis}.

\subsection{Eigenvalue decomposition}
Here we review the technology laid out in \cite{Karlovini:2002fc, Andersson:2006ze, Brito:2009jj} for dealing with isotropic elastic solids. The main point is to identify the maximum number of eigenvalues and eigenvectors, and to use a common eigenvector basis with which space-time and  material quantities can be expanded. In this section we make no assumption about symmetries like  spherical, axial, or static; in the next section we will, and consequently a lot of the expressions simplify from the general case.

The eigenvalues of the pulled-back material metric ${k^a}_b$ are written as $n^2_{\mathbb{A}}$, where $\mathbb{A} = 1, 2, 3$ label each of the ``principle'' directions. The particle density $n$ is given in terms of the eigenvalues as
\bea
n = n_1n_2n_3,
\eea
which are interpretable as the principle linear particle densities.
In an orthonormal basis $\{{e^a}_{\mathbb{A}}\}$ it follows that the space-time metric decomposes as
\bea
g_{ab} = - u_au_b + \sum_{\mathbb{A}=1}^3e_{a\mathbb{A}}e_{b\mathbb{A}},
\eea
and the pulled-back material metric decomposes as
\bea
k_{ab} =  \sum_{\mathbb{A}=1}^3n_{\mathbb{A}}^2e_{a\mathbb{A}}e_{\mathbb{A}}.
\eea
Hence, $\pd{}{g^{ab}}$ acting on a quantity $X$ which is a function of scalar invariants of ${k^a}_b$, is
\bea
\pd{X}{g^{ab}} = \half  \sum_{\mathbb{A}=1}^3e_{a\mathbb{A}}e_{b\mathbb{A}}n_{\mathbb{A}}\pd{X}{n_{\mathbb{A}}}.
\eea
Using this, the pressure tensor is given by
\bea
P_{ab} =  \sum_{\mathbb{A}=1}^3P_{\mathbb{A}}e_{a\mathbb{A}}e_{b\mathbb{A}},
\eea
in which the principle values of the pressure tensor are given by
\bea
P_{\mathbb{A}} = nn_{\mathbb{A}}\pd{\epsilon}{n_{\mathbb{A}}}.
\eea
Using ${k^a}_b = n^{2/3}{\eta^a}_b$ naturally splits the pressure tensor into the pressure scalar and anisotropic pressure as we now show. Denoting the eigenvalues of ${\eta^a}_b$ as $\alpha_{\mu}^2$, and are related to the $n_{\mu}$ via
\bea
\alpha_{\mathbb{A}} = \frac{n_{\mathbb{A}}}{n^{1/3}} = \left(\frac{z_{\mathbb{A}+2}}{z_{\mathbb{A}+1}} \right)^{1/3},
\eea
in which
\bea
z_{\mathbb{A}} = \frac{n_{\mathbb{A}+1}}{n_{\mathbb{A}+2}} = \frac{\alpha_{\mathbb{A}+1}}{\alpha_{\mathbb{A}+2}}.
\eea 
It follows that
\bse
\bea
n_{\mathbb{A}}\pd{}{n_{\mathbb{A}}} = n\pd{}{n} + z_{\mathbb{A}+2}\pd{}{z_{\mathbb{A}+2}} - z_{\mathbb{A}+1}\pd{}{z_{\mathbb{A}+1}},
\eea
and furthermore that
\bea
\eta_{c\langle a}\pd{}{{\eta^{b\rangle}}_c} = \half  \sum_{\mathbb{A}=1}^3e_{a\mathbb{A}}e_{b\mathbb{A}}\left( z_{\mathbb{A}+2}\pd{}{z_{\mathbb{A}+2}} - z_{\mathbb{A}+1}\pd{}{z_{\mathbb{A}+1}}\right).
\eea
\ese 
The principle pressures are given by the sum
\bse
\bea
p_{\mathbb{A}} = p + \pi_{\mathbb{A}},
\eea
where the eigenvalues $\pi_{\mathbb{A}}$ are
\bea
\pi_{\mathbb{A}} = n\left( z_{\mathbb{A}+2}\pd{\epsilon}{z_{\mathbb{A}+2}} - z_{\mathbb{A}+1}\pd{\epsilon}{z_{\mathbb{A}+1}}\right),
\eea
\ese
which satisfy
\bea
\sum_{\mathbb{A}=1}^3\pi_{\mathbb{A}}=0.
\eea
In terms of the $\pi_{\mathbb{A}}$, the anisotropic pressure tensor is 
\bea
\pi_{ab} =  \sum_{\mathbb{A}=1}^3\pi_{\mathbb{A}}e_{a\mathbb{A}}e_{b\mathbb{A}}.
\eea


\subsection{Static spherically symmetric configurations}
We continue to review \cite{Karlovini:2002fc, Andersson:2006ze, Brito:2009jj}, and use the technology outlined in the previous section to construct the relevant equations to describe static spherically symmetric configurations. 

The metric for static spherically symmetric space-time   decomposes as
\bse
\label{gen-sph-summ-schem-decomp-fldhfkd}
\bea
g_{ab} = - u_au_b +\gamma_{ab},
\eea
where the orthogonal metric $\gamma_{ab}$ splits up into a ``radial'' vector and ``angular'' metric via
\bea
\label{gen-sph-summ-schem-decomp-fldhfkd-gamma}
\gamma_{ab} =  r_ar_b + t_{ab}.
\eea
The velocity vector $u_a$, radial vector $r_a$, and totally orthogonal metric $t_{ab}$ are given by
\bea
u_a = - e^{\nu(r)}\left( \dd t\right)_a,\qquad r_a = e^{\lambda(r)}\left( \dd r\right)_a,\qquad t_{ab} = r^2\left( \dd\Omega^2\right)_{ab}
\eea
and $\lambda(r)$ is specified by the ``Schwarzschild mass'' function $m(r)$ via
\bea
e^{-2\lambda(r)} = 1 - \frac{2m(r)}{r}.
\eea
\ese
The gravitational field equations set the Einstein tensor  equal to the usual form of the solid energy-momentum tensor 
\bse
\label{gen-sph-summ-schem-decomp-fldhfkd-12}
\bea
T_{ab} = \rho u_au_b + P_{ab},
\eea
in which the only  pressure tensor compatible with the spherical symmetry decomposes as
\bea
P_{ab} = \qsubrm{p}{r}r_ar_b + \qsubrm{p}{t}t_{ab}.
\eea
\ese
One should interpret $\qsubrm{p}{r}$ as the radial pressure, and $\qsubrm{p}{t}$ as the tangential pressure. 

In all generality  (i.e., for any static spherically symmetric configuration) the Einstein equations ${G^a}_b = \kappa {T^a}_b$ for the metric (\ref{gen-sph-summ-schem-decomp-fldhfkd}) and energy-momentum tensor (\ref{gen-sph-summ-schem-decomp-fldhfkd-12}) are given by
\bse
\bea
\frac{\dd \nu}{\dd r} = \frac{m + \tfrac{1}{2}\kappa r^3\qsubrm{p}{r}}{r\left(r-2m\right)},
\eea
\bea
\label{gen-ela-feb}
\frac{\dd m}{\dd r} = \half \kappa r^2\rho,
\eea
\bea
\label{gen-ela-fec}
\frac{\dd \qsubrm{p}{r}}{\dd r} = - \left( \rho + \qsubrm{p}{r}\right) \frac{m + \tfrac{1}{2}\kappa r^3\qsubrm{p}{r}}{r\left(r-2m\right)} + 6\frac{q}{r},
\eea
\ese
where the difference between the radial and tangential pressures is quantified via 
\bea
q \defn \tfrac{1}{3}\left(\qsubrm{p}{t} -\qsubrm{p}{r}\right).
\eea
Since the metric variable $\nu$ does not appear in (\ref{gen-ela-feb}) or (\ref{gen-ela-fec}), we do not need to consider it in what follows, if all we are interested in is the profiles of the elastic matter fields.

The crucial part of making the source to the field equations that due to an elastic solid, is to compute $\rho, \qsubrm{p}{r}$, and $\qsubrm{p}{t}$ from the equation of state. For that we must decompose the material metric, and pull it back to space-time. Symmetry dictates that the equation of state $\epsilon$ has only two arguments,
\bea
\epsilon = \epsilon(\qsubrm{n}{r},\qsubrm{n}{t}),
\eea
and the particle number density $n$ is given by
\bea
\label{eq:sec:n-nr-nt-decomp}
n = \qsubrm{n}{r}\qsubrm{n}{t}^2.
\eea
Note that $\qsubrm{n}{r}$ and $\qsubrm{n}{t}$ are interpretable as the radial and tangential linear number densities.
The material metric must have  symmetries similar to the space-time metric to preserve isotropy; in the material space the material metric takes on the form
\bea
k_{AB} = \tilde{r}_A\tilde{r}_B + \tilde{t}_{AB}.
\eea
The basis vector and tangential tensor in the material manifold are given by
\bea
\tilde{r}_A = e^{\tilde{\lambda}}\left( \dd\tilde{r}\right)_A,\qquad \tilde{t}_{AB} = \tilde{r}^2\left( \dd\tilde{\Omega}^2\right)_{AB}.
\eea
The pulled back material metric is given by
\bea
\label{eq:nr-nt-kab-defn}
k_{ab} = \qsubrm{n}{r}^2r_ar_b + \qsubrm{n}{t}^2t_{ab},
\eea
where the two space-time components of the pulled-back material metric (\ref{eq:nr-nt-kab-defn}) are given by
\bea
\qsubrm{n}{r} = e^{\tilde{\lambda} - \lambda}\frac{\dd \tilde{r}}{\dd r},\qquad \qsubrm{n}{t} = \frac{\tilde{r}}{r}.
\eea
The mapping between the material and space-time manifolds is thus defined through the relationship between the radial coordinate in the material manifold, and that in space-time $\tilde{r} = \tilde{r}(r)$: this is entirely encapsulated by the two space-time functions $\qsubrm{n}{r}$ and $\qsubrm{n}{t}$. The constant volume shear tensor, defined in (\ref{eq:sec:cons-vol-shear-tenor}) is given by
\bea
s_{ab} = \tfrac{1}{2}\left( \gamma_{ab} - n^{-2/3}k_{ab} \right),
\eea
where we used (\ref{eq:sec:k-eta-n-pb}) to replace the uni-modular tensor $\eta_{ab}$ with $n$ and $k_{ab}$. Using (\ref{gen-sph-summ-schem-decomp-fldhfkd-gamma}), (\ref{eq:sec:n-nr-nt-decomp}), and (\ref{eq:nr-nt-kab-defn}) we thus obtain
\bea
s_{ab} = \tfrac{1}{2} \left( \left[ 1 - \left( \frac{\qsubrm{n}{r}}{\qsubrm{n}{t}}\right)^{4/3} \right]r_ar_b + \left[1 -  \left( \frac{\qsubrm{n}{t}}{\qsubrm{n}{r}}\right)^{4/3} \right]t_{ab} \right).
\eea
Hence, we observe that $s_{ab}=0$ when, and only when, the radial and tangential number densities are identical:
\bea
s_{ab} = 0\qquad \Longleftrightarrow\qquad  \qsubrm{n}{r} = \qsubrm{n}{t}.
\eea
Thus, $s_{ab}=0$ when the solid ``becomes'' a fluid.

Rather than work with $n_r$ and $n_t$, it is useful to work with the combinations
\bse
\label{eq:defn-soh-sym-n-z}
\bea
n \defn \qsubrm{n}{r}\qsubrm{n}{t}^2 = \left(\frac{\tilde{r}}{r} \right)^3z,
\eea
\bea
\label{eq:defn-soh-symz}
z \defn \frac{\qsubrm{n}{r}}{\qsubrm{n}{t}} = e^{\tilde{\lambda} - \lambda} \frac{r}{\tilde{r}}\frac{\dd\tilde{r}}{\dd r}.
\eea
\ese
One should keep in mind that $n$ is the particle number density. Also, note that $z=1$ when $\qsubrm{n}{r}=\qsubrm{n}{t}$: this is the fluid limit. In terms of $(n,z)$ as defined in (\ref{eq:defn-soh-sym-n-z}) the field equations (\ref{gen-ela-feb}, \ref{gen-ela-fec}) can be written as
\bse
\label{eq:fiedeqns-elast-general-1}
\bea
\frac{\dd m}{\dd r}= \half \kappa r^2\rho,
\eea
\bea
\frac{\dd n}{\dd r} = \frac{n}{r\qsubrm{\beta}{r}}\left[ - \left( \rho + \qsubrm{p}{r}\right) \frac{m+\frac{1}{2}\kappa r^3\qsubrm{p}{r}}{r-2m} + 6q + 3z\pd{\qsubrm{p}{r}}{z}\left( ze^{\lambda - \tilde{\lambda}}-1\right) \right],
\eea
\bea
\frac{\dd z}{\dd r} = z\left[\frac{1}{n}\frac{\dd n}{\dd r} - \frac{3}{r}\left( ze^{\lambda - \tilde{\lambda}}-1\right) \right],
\eea
\ese
in which $\tilde{\lambda}$ parameterizes the curvature of the material manifold (which can   be set to zero), and where $\left\{\rho, \qsubrm{p}{r},q,\qsubrm{\beta}{r}\right\}$ are given in terms of the two-parameter equation of state $\epsilon = \epsilon(n,z)$ via
\bse
\bea
\rho = n \epsilon,\qquad
\qsubrm{p}{r} = n^2\pd{\epsilon}{n}-2q,
\eea
\bea
q= - \half nz\pd{\epsilon}{z},\qquad
\qsubrm{\beta}{r} = n \pd{\qsubrm{p}{r}}{n}+z\pd{\qsubrm{p}{r}}{z}.
\eea
\ese
Explicitly,
\bea
\qsubrm{\beta}{r} = n \left( 2n\pd{\epsilon}{n}+n^2\pd{^2\epsilon}{n^2} + 2 z\pd{\epsilon}{z} + 2 nz\pd{^2\epsilon}{n\partial z} + z^2\pd{^2\epsilon}{z^2} \right).
\eea

By way of  a ``fiducial'' example, consider the quasi-Hookean ansatz for the equation of state,
\bea
\epsilon = \check{\epsilon} + \frac{1}{n}\check{\mu}s^2,
\eea
where a quantity with an overhead ``check'' symbol denotes that the quantity only depends on $n$, and
\bea
s^2= \tfrac{1}{6}\left(z^{-1}-z\right)^2
\eea 
is the shear scalar we defined in (\ref{eq:sec:s2-choice-1}), although there we used the symbol $\overline{s}^2$, and we have now expressed it in terms of $z$ as defined in (\ref{eq:defn-soh-symz}). Note that in the fluid limit $z=1$, the shear scalar $s^2=0$. The Einstein equations (\ref{eq:fiedeqns-elast-general-1}) can be formulated in the independant variables $\left(m,\check{p},z\right)$, and become
\bse
\bea
\frac{\dd m}{\dd r} = \half \kappa r^2\rho,
\eea
\bea
\frac{\dd\check{p}}{\dd r} = \frac{\check{\beta}}{r\qsubrm{\beta}{r}}\left[- \left( \rho + \qsubrm{p}{r}\right) \frac{m+\half \kappa r^3 \qsubrm{p}{r}}{r-2m} + 6q + 4 \left(ze^{\lambda - \tilde{\lambda}}-1\right) \left(\check{\mu} + 3 \sigma + \tfrac{3}{2}\left(1-\check{\Omega}\right)q \right) \right],\nonumber\\
\eea
\bea
\frac{\dd z}{\dd r} = \frac{z}{r}\left[ \frac{r}{\check{\beta}}\frac{\dd \check{p}}{\dd r} - 3\left(ze^{\lambda - \tilde{\lambda}}-1\right)  \right],
\eea
\ese
in which
\bse
\bea
\sigma = \check{\mu}s^2,\qquad q = \check{\mu}\chi,\qquad \rho = \check{\rho} + \sigma,
\eea
\bea
 \qsubrm{p}{r} = p - 2 q,\qquad p = \check{p} + \left( \check{\Omega}-1\right)\sigma,
\eea
\bea
\chi = \tfrac{1}{6}\left(z^{-2}-z^2\right),\qquad \check{\beta} = \left( \check{\rho} + \check{p}\right) \frac{\dd \check{p}}{\dd\check{\rho}},\qquad \check{\Omega} = \frac{\check{\beta}}{\check{\mu}}\frac{\dd \check{\mu}}{\dd\check{p}},
\eea
\bea
\qsubrm{\beta}{r} = \beta+4\left[ \sigma + \left( \check{\Omega} - \half \right) \right],\qquad \beta = \check{\beta} + \frac{4}{3}\check{\mu} + \left[ \check{\Omega} \left( \check{\Omega} - 1 \right) + \check{\beta} \frac{\dd \check{\Omega}}{\dd\check{p}}\right].
\eea
\ese
One now requires two functions of state to complete the system of equations: $\check{\rho}\left( \check{p}\right)$ and $\check{\mu}\left( \check{p}\right)$. One may be slightly uncomfortable with this idea: if that is the case, then one should recall that in the more familiar ``scalar field models'' one needs to pick parameterizations of the Lagrangian density (even the canonical theory has one free function, the potential $V(\phi)$). 
An explicit relationship between $\check{\rho}$, $\check{\mu}$ and $\check{p}$ is provided by 
\bea
\check{\rho} = \frac{\qsubrm{p}{c}}{\check{\Gamma}-1} \left[ w\frac{\check{p}}{\qsubrm{p}{c}} + \left( \frac{\check{p}}{\qsubrm{p}{c}}\right)^{1/\check{\Gamma}}\right],\qquad \check{\mu} = k \check{p}.
\eea
This is a modified polytropic equation of state. The parameter $\qsubrm{p}{c}$ indicates the pressure scale at which the transition between the linear and polytropic behavior occurs. The parameter $k$ quantifies the rigidity, and vanishes in the fluid limit.

\subsubsection{My spherical symmetry calculations}
Here we use the metric
\bea
\label{eq:metric-sph-symm-gen}
\dd s^2 = - e^{2\nu(r)} \dd t^2 + \left( 1 - \frac{2m(r)}{r}\right)^{-1} \dd r^2 + r^2\dd\Omega^2,
\eea
and the total energy-momentum tensor
\bea
{T^a}_b = \diag\left( -\rho(r),  \qsubrm{p}{r}(r), \qsubrm{p}{t}(r), \qsubrm{p}{t}(r)\right).
\eea
The components are: energy density $\rho$, radial pressure $ \qsubrm{p}{r}$, and tangential pressure $ \qsubrm{p}{t}$. One should bear in mind that these components could be formed from multiple constituents (although we will come back to that case later on).


The non-zero components of the Einstein tensor  computed from the metric (\ref{eq:metric-sph-symm-gen}) are given by
\bse
\bea
\label{my-G00}
{G^0}_0 = -\frac{2 m'}{r^2},
\eea
\bea
\label{my-Grr}
{G^r}_r = -\frac{2}{r^3} \left(m+r\left(2  m-r\right) \nu '\right)
\eea
\bea
{G^{\theta}}_{\theta}  = {G^{\phi}}_{\phi}&=& \frac{1}{r^3}\bigg(-m \left(-1+r \nu '+2 r^2 \nu '^2+2 r^2 \nu ''\right)\nonumber\\
&&+r \left(-m' \left(1+r \nu '\right)+r \left(\nu '+r \nu '^2+r \nu ''\right)\right)\bigg).
\eea
\ese



The 00- and $rr$-Einstein field equations ${G^a}_b = \kappa {T^a}_b$ thus read
\bse
\label{eq:gen-TVO-aniso}
\bea
\label{eq:gen-TVO-aniso-1}
m' = \tfrac{1}{2}\kappa r^2\rho,
\eea
\bea
\nu' = \frac{m+\half \kappa r^3\qsubrm{p}{r}}{r\left(r-2m\right)}.
\eea
The only non-zero entry of the conservation equation $\nabla_a{T^a}_b=0$ is
\bea
\label{eq:gen-TVO-aniso-3}
\qsubrm{p}{r}' = - \left( \rho + \qsubrm{p}{r}\right) \frac{m+\half \kappa r^3\qsubrm{p}{r}}{r\left(r-2m\right)} + \frac{2}{r}\left(\qsubrm{p}{t} - \qsubrm{p}{r}\right).
\eea
\ese
The system of three equations (\ref{eq:gen-TVO-aniso}) constitute the \textit{generalized Tolman-Oppenheimer-Volkov} equations, where a star is constructed from an anisotropic fluid. Notice that there are three equations for five unknowns (i.e., $m(r), \nu(r), \rho(r), \qsubrm{p}{r}(t)$, and $\qsubrm{p}{t}(r)$). And so, two ``equations of state'' must be given: these are typically taken to relate the pressures to the density. Note that the metric variable $\nu$ does not appear in (\ref{eq:gen-TVO-aniso-1}) or (\ref{eq:gen-TVO-aniso-3}): it does not need to be solved for at the same time as $m$ and $\qsubrm{p}{r}$ (this is statement which helps computations).

There are some simple circumstances under which the generalized TOV equations become analytically soluble \cite{1974ApJ...188..657B}. When the density is taken to be constant throughout the star, $\rho = \rho_0$, it is simple to integrate (\ref{eq:gen-TVO-aniso-1}) to obtain 
\bea
m(r) = \frac{4\pi G}{3}\rho_0 r^3.
\eea
 We now assume an equation of state (which is the final piece of information required to close the system of equations) in the form
\bse
\bea
\qsubrm{p}{t} = \qsubrm{p}{r} + Af(\qsubrm{p}{r},r) \left( \rho + \qsubrm{p}{r}\right)r^n,
\eea
with the function $f$ given by
\bea
f(\qsubrm{p}{r},r) = \frac{\rho + 3 \qsubrm{p}{r}}{1 - 2m/r},
\eea
\ese
and where there are two free parameters, $A$ and $n$. Using this,  (\ref{eq:gen-TVO-aniso-3}) can be written as
\bea
\qsubrm{p}{r}' = - \left( \rho_0 + \qsubrm{p}{r}\right) \left( \rho_0 + 3\qsubrm{p}{r}\right)  \frac{ \frac{4\pi G}{3} - 2Ar^{n-2}}{1-\frac{8\pi G \rho_0}{3}r^2}r.
\eea
When $n=2$, this can be integrated to give
\bea
\qsubrm{p}{r}(r) = \rho_0 \left[ \frac{\left( 1-2m/r \right)^Q - \left(1 - 2M/R\right)^Q}{3\left(1-2M/R\right)^Q - \left( 1 - 2m/r\right)^Q}\right]
\eea
with
\bea
\qsubrm{p}{r}(r=R)=0,\qquad m(R) = M,\qquad Q = \half - \frac{3A}{4\pi G}.
\eea
The pressure at the centre of the configuration is given by
\bea
\qsubrm{p}{c} = \rho_0\frac{1 - \left(1-2M/R\right)^Q}{3\left( 1 - 2M/R\right)^Q-1}.
\eea
The critical combination of $M$ and $R$ for which the central pressure becomes infinite is found  by noting when the denominator of $\qsubrm{p}{c}$ becomes zero; i.e., when
\bea
\qsubrm{\left(2M/R\right)}{crit} = 1- \left(\frac{1}{3}\right)^{1/Q} .
\eea
One requires $\qsubrm{\left(2M/R\right)}{crit}\geq 0$ for physical systems. This is known as \textit{Buchdahl's inequality} \cite{PhysRev.116.1027, 1964ApJ...140..434T} (although it also exists for more general setups than that which we are considering here). In the perfect-fluid case there is no anisotropy, $A = 0$, and so $Q= 1/2$, and so
\bea
\qsubrm{\left(2M/R\right)}{crit} = 8/9.
\eea
When $A = 2\pi G /3$, one has $Q=0$ and thus
\bea
\qsubrm{\left(2M/R\right)}{crit} = 1,
\eea
which is the Schwarzschild value.

Note that 
\bea
\sqrt{-g} = \left(1-\frac{2m}{r}\right)^{-1/2}e^{\nu} r^2\sin\theta.
\eea
The total mass in a volume is given by
\bea
M \defn \int \dd^3x\, \sqrt{-g}\, m(r) 
\eea

\subsection{Stars immersed in an elastic solid}
It should be clear from our presentation so far that an elastic solid is an anisotropic fluid. In \cite{1974ApJ...188..657B, Dev:2000gt, Mak:2001eb, Dev:2003qd, Das:2003wm} anisotropic stars are studied. See also \cite{Gorini:2008zj, Gorini:2009em}, which study stars (i.e., solutions to the TOV equations) immersed in Chaplygin-gas; also, \cite{Folomeev:2013hoa} study stars in a Chameleon scalar-field background. Also, \cite{Dzhunushaliev:2014mza} look at stars in anisotropic fluids, but in the context of wormholes. In non of these papers is any mention made of the link between anisotropy and elasticity.

We would like to understand
\begin{enumerate}
\item Stars in vacuum
\item Stars in fluids
\item Stars in solids
\end{enumerate}
Point 1. is about solving the Tolman-Oppenheimer-Volkov equations (usually studied with a model neutron-star equation of state). The second is about solving the TOV equations, but in a non-empty space-time.

\subsubsection{Example model}
A simple example of a model of a star immersed in a ``cosmological'' background is
\bea
S = \int \dd^4x\,\sqrt{-g} \, \left[ \frac{R}{16\pi G} + \qsubrm{\ld}{star}+\qsubrm{\ld}{cosm} \right],
\eea
wherein $R$ is the Ricci scalar,  $\qsubrm{\ld}{star}$ is the Lagrangian density of the matter which gives rise to the ``star'', and $\qsubrm{\ld}{cosm}$ is the Lagrangian density of the ``cosmological'' dark energy. The field equations are given by
\bea
{G^a}_b = \kappa \left( {T^a}_b + {U^a}_b\right),
\eea
in which ${T^a}_b$ is the energy-momentum tensor for the ``star Lagrangian'', and ${U^a}_b$ is the energy-momentum tensor for the cosmological dark energy (to be in-keeping with some of our other work, we call ${U^a}_b$ the dark energy-momentum tensor). If the model is minimally coupled then the two energy-momentum tensors are separately conserved; else, they have a coupling current which transfers energy between the cosmological and stellar matter fields.

Typical examples of the dark energy Lagrangians are
\bea
\qsubrm{\ld}{cosm} \in \left\{ \begin{array}{c} \mbox{scalar field} , \\ \mbox{solid} , \\ \mbox{perfect fluid}. \end{array}\right. 
\eea
It is to be remarked that the perfect fluid is a particular limit of the solid (a solid with vanishing rigidity is a perfect fluid).
In each case, the components of the dark energy-momentum tensor   are given respectively by
\bea
 {U^a}_b\in \left\{ \begin{array}{c}     \partial^a\phi\partial_b\phi +{g^a}_b\left(\tfrac{1}{2}\partial_c\phi\partial^c\phi - V(\phi) \right), \\    \rho u^au_b + p{\gamma^a}_b + {\pi^a}_b,  \\  \rho u^au_b + p{\gamma^a}_b.  \end{array}\right.
\eea

The total source to the gravitional field equations can be written as
\bea
{\qsubrm{T}{tot}^a}{}_b = {T^a}_b + {U^a}_b.
\eea
This will engender the following decomposition for the total energy density $\qsuprm{\rho}{tot}$,   the radial pressure $\qsupbrm{p}{tot}{r}$, and tangential pressure $\qsupbrm{p}{tot}{t}$ components,
\bse
\bea
\qsuprm{\rho}{tot} = \qsuprm{\rho}{m} + \qsuprm{\rho}{d},
\eea
\bea
\qsupbrm{p}{tot}{r}= \qsupbrm{p}{m}{r} + \qsupbrm{p}{d}{r}
\eea
\bea
\qsupbrm{p}{tot}{t} = \qsupbrm{p}{m}{t} + \qsupbrm{p}{d}{t}
\eea
\ese
Components with an ``m'' super-script label correspond to the component of the stellar energy-momentum tensor, and those with a ``d'' super-script to the components of the dark energy-momentum tensor. This will mean that the field equations (in the minimally coupled case) are given by an appropriately modified version of (\ref{eq:gen-TVO-aniso}). To be specific,
\bse
\bea
\nu' = \frac{m+\half \kappa r^3\qsupbrm{p}{tot}{r}}{r\left(r-2m\right)},
\eea
\bea
\label{mult-comp-a}
m' = \half \kappa r^2\qsuprm{\rho}{tot},
\eea
\bea
\label{mult-comp-c}
\qsupbrm{p}{i}{r}{}' = -  \left( \qsuprm{\rho}{i} + \qsupbrm{p}{i}{r}\right) \frac{m+\half \kappa r^3\qsupbrm{p}{tot}{r}}{r\left(r-2m\right)} + \frac{2}{r}\left( \qsupbrm{p}{i}{t}- \qsupbrm{p}{i}{r} \right),
\eea
\ese
where there is a copy of (\ref{mult-comp-c}) for each ${\rm i} \in \left\{ \rm{m}, \rm{d}\right\}$.
For a perfect  fluid model of the star, the matter components satisfy
\bea
\qsupbrm{p}{m}{r}= \qsupbrm{p}{m}{t},\qquad \qsupbrm{p}{m}{r}(R_0) = 0,\qquad \qsuprm{\rho}{m}(r>R_0)=0.
\eea
That is, the radial and tangential pressures are identical,  the pressure vanishes at the radius $R_0$, which is supposed to be the surface of the star, and the stellar density vanishes outside of the star.


Integrating (\ref{mult-comp-a}) from $r = 0$ to $r = R_{\infty}$ yields a total mass $M_{\infty}$, which can be broken up into three contributions as
\bea
M_{\infty} =  \qsuprm{M}{m}_{\star} + \qsuprm{M}{d}_{\star} + \qsuprm{M}{d},
\eea
where
\bse
\bea
\qsuprm{M}{m}_{\star} \defn \half \kappa\int_0^{R_0} \dd r \, r^2\, \qsuprm{\rho}{m},
\eea
\bea
\qsuprm{M}{d}_{\star} \defn \half \kappa\int_0^{R_0} \dd r \, r^2\, \qsuprm{\rho}{d},
\eea
\bea
\qsuprm{M}{d} \defn \half \kappa\int_{R_0}^{R_{\infty}} \dd r \, r^2\, \qsuprm{\rho}{d}
\eea
\ese
These can be interpreted as the mass of the star, the mass of the dark substance inside the star, and the mass of the dark substance outside of the star (i.e., in the cosmology).

\cleardoublepage
\section{Mixing solids, fluids, and scalar fields}
We may also be interested in composite descriptions where there are solids interacting with scalar fields or fluids.
\subsection{Solids and fluids}
We are interested in constructing the action for a fluid coupled to a solid. It is useful to note that in Alkistis et al \cite{Pourtsidou:2013nha} consider a fluid coupled to a scalar field.

See \cite{Langlois11071998}, \cite{Haskell:2012vp}, \cite{Andersson:2005pf}, \cite{Andersson:2006nr}.

Suppose one had two fluids; each fluid have currents $a^{\mu}$ and $b^{\mu}$. The allowed invariants are
\bea
a\defn (-a^{\mu}a_{\mu})^{1/2},\qquad x \defn (-a^{\mu}b_{\mu})^{1/2},\qquad b = (-b^{\mu}b_{\mu})^{1/2}.
\eea
\subsection{Solids and scalar fields}
The theory of a solid which we have considered in this review was constructed from all invariants of the elements of the set $\{n, u^a, {\eta^a}_b\}$. In \cite{Pourtsidou:2013nha} a similar question was asked, but for a fluid: they asked for the general theory of a scalar+fluid mixture; thus, they found all invariants formed out of the relevant fields, which are $\{n, u^a,\phi, \nabla_a\phi\}$. There are at most four scalar invariants, and so the Lagrangian for a scalar+fluid mixture has at most four scalar arguments, and can be written as
\bea
\qsubrm{\ld}{s+f} =\qsubrm{\ld}{s+f}\left( n, \phi, u^a\nabla_a\phi,\nabla_a\phi\nabla^a\phi\right).
\eea

 Suppose we now had a scalar field in the matter action: we now want to form all invariants from the set $\{n, u^a, {\eta^a}_b, \phi, \nabla_a\phi\}$. See Section \ref{sec:hyperelasticity} for a discussion on hyper-elasticity which may be more helpful.

For the solid+scalar mixture, there is a tower of invariants. For example, the first few are
\bea
\bigg\{ \phi, \quad n,\quad u^a\nabla_a\phi,\quad\nabla^a\phi\nabla_a\phi, \quad [\gbm{\eta}],\quad [\gbm{\eta}^2],\quad {\eta^a}_b\nabla_a\phi\nabla^b\phi,\quad {\eta^a}_b{\eta^b}_c\nabla_a\phi\nabla^c\phi,\quad \ldots\bigg\}\nonumber\\
\eea
The first four above are present in the scalar+fluid mixture: the rest are only present in the scalar+solid mixture. To aid the construction of allowed invariants, it is useful to define the quantity
\bea
\label{eq:sec:Cabc-defn-mix}
{C^a}_b{}^c \defn {\eta^a}_b\nabla^c\phi.
\eea
Note that $u^b{C^a}_b{}^c = 0$; i.e., the tensor ${C^a}_b{}^c$ is orthogonal on its last two indices.
The second useful step  in cataloguing the terms is by keeping track of the number of space-time derivatives. This is done by replacing $x^a\rightarrow Mx^a$, and keeping track of the powers of $M$. First of all, note that the uni-modular tensor has 2-powers of $M$,
\bse
\bea
\eta_{ab} = n^{-2/3} k_{AB} \partial_a\phi^A\partial_b\phi^B \sim \frac{1}{M^2} 
\eea
The rank-3 tensor defined in (\ref{eq:sec:Cabc-defn-mix}) carries 3 powers of $M$,
\bea
{C^a}_b{}^c  \sim \frac{1}{M^3} ,
\eea
the $\qsuprm{n}{th}$-trace of $\gbm{\eta}$ carries $2n$-powers of $M$,
\bea
[\gbm{\eta}^n] \sim \frac{1}{M^{2n}},
\eea
and the canonical kinetic term for the scalar carries 2 powers of $M$,
\bea
2\kin \defn \partial_a\phi\partial^a\phi \sim \frac{1}{M^2}.
\eea
In addition, the contraction of the scalar field derivative with the time-like unit vector carries a single power of $M$,
\bea
\mathcal{Z} \defn u^a\partial_a\phi \sim \frac{1}{M}.
\eea
\ese

Using these countings, some of the first few invariants that can be formed from the solid+scalar mixture are
\bse
\bea
{C^{ab}}_b\partial_a\phi = 2 \kin [\gbm{\eta}] \sim \frac{1}{M^4},
\qquad
{C^{ba}}_b \partial_a\phi= \eta^{ab}\partial_a\phi\partial_b\phi \sim\frac{1}{M^4},
\eea
\bea
C^{abc}C_{abc} = 2\kin [\gbm{\eta}^2] \sim \frac{1}{M^6},
\qquad
C^{abc}C_{bac} = \eta^{ac}{\eta^b}_c \partial_a\phi\partial_b\phi \sim\frac{1}{M^6}.
\eea
\ese
Of course, this list is not exhaustive, but it nicely illustrate how some of the invariants constructed using $C$ could be related to each other.
The Lagrangian, built out of such considerations, will be of the form
\bea
\ld &=& \frac{1}{M}c_{1,1}\mathcal{Z}\nonumber\\
&&+ \frac{ 1}{M^2}\left(c_{2,1}\mathcal{Z}^2 + c_{2,2}\kin + c_{2,3}[\gbm{\eta}]\right) \nonumber\\
&&+\frac{1}{M^3}\left(c_{3,1}\mathcal{Z}^3 + c_{3,2} \mathcal{Z}\kin + c_{3,3}\mathcal{Z}[\gbm{\eta}]\right) \nonumber\\
&&+ \frac{1}{M^4} \left(c_{4,1} \mathcal{Z}^4  + c_{4,2}\mathcal{Z}^2\kin + c_{4,3}\mathcal{Z}^2[\gbm{\eta}]+ c_{4,4}\kin^2 \right.\nonumber\\
&&\left.\qquad\qquad+ c_{4,5}\kin [\gbm{\eta}] + c_{4,6}[\gbm{\eta}]^2+ c_{4,7}[\gbm{\eta}^2] + c_{4,8} \eta^{ab}\partial_a\phi\partial_b\phi\right) + \ldots
\eea
where the coefficients $c_{I,J} = c_{I,J}\left(n, \phi\right)$ are explicitly   constructed without any derivatives, and the ellipses stand for terms in higher powers of $M$.

Then we can form
\bea
\mathcal{I} = \bigg\{ {C^a}_a{}^c\nabla_c\phi,\qquad\ldots \bigg\}.
\eea

We cannot (or at least, we are not aware of how to) construct all invariants out of these fields. Infact, it does not seem that this construction is the most elegant one can envisage -- for that one should turn to our review of hyper-elasticity, which we save for the next subsection. However, we can still obtain field equations. Let us begin from a statement about the field content of the scalar+solid theory
\bea
\ld = \ld(g_{ab},n^a, {\eta^a}_b,\phi, \partial_a\phi).
\eea
In a cosmological context, one can isolate the projections of the relevant quantities, at the level of the  background:
\bse
\label{cosm-bg-info-mix-1}
\bea
g_{ab} = \gamma_{ab} - u_au_b,\qquad \partial_a\phi = - \dot{\phi}u_a,\qquad n^a = n u^a,\qquad {\eta^a}_b = {\gamma^a}_c{\gamma_b}^d {\eta^c}_d.
\eea
Note that by orthogonality
\bea
{\eta^a}_b\partial_a\phi= n^a{\eta^b}_a=0.
\eea
Also, on the background, the uni-modular tensor ${\eta^a}_b$ is strictly isotropic, and so completely decomposes as
\bea
{\eta^a}_b = \omega{\gamma^a}_b.
\eea
Thus,
\bea
[\gbm{\eta}] = 3\omega,\qquad [\gbm{\eta}^2] = 3\omega^2.
\eea
\ese
By combining (\ref{cosm-bg-info-mix-1}) we find that on the cosmological background there are only 5 scalar invariants, and so the Lagrangian for the background is a function with 5 arguments:
\bea
\ld = \ld\left(\phi, n, \omega, u^a\partial_a\phi, \partial^a\phi\partial_a\phi\right).
\eea



The variation in the function $\ld$ is given by
\bea
\delta \ld = \pd{\ld}{g_{ab}}\lp g_{ab} + \pd{\ld}{n^a}\lp n^a + \pd{\ld}{{\eta^a}_b} \lp {\eta^a}_b + \pd{\ld}{\phi}\lp \phi + \pd{\ld}{\partial_a\phi}  \partial_a\lp\phi.
\eea
Since
\bea
\lp n^a = - \tfrac{1}{2}n^ag^{cd}\lp g_{cd},
\eea
one can find
\bea
\ep n^a = - \tfrac{1}{2}n^ag^{cd}\ep g_{cd} + n^b\nabla_b\xi^a - \nabla_b\left( n^a\xi^b\right)
\eea
Note that 
\bea
\ld = \ld(\phi, \partial_a\phi, {k^a}_b)
\eea

\subsection{Hyper-elasticity as a road to mixing scalars and solids}
\label{sec:hyperelasticity}
The idea of hyper-elasticity \cite{Carter:2006cw} may give an elegant insight as to how to incorporate general coupling between a solid and a scalar field.
\subsubsection{Hyper-elasticity}
Suppose the theory is constructed from a Lagrangian density $\ld$ formed as a general function of $({\rm p}+1)$-scalar fields and their space-time gradients,
\bea
\label{eq:sec:lag-gen-hyper-nsjk}
\ld = \ld \left(\phi^0, \phi^1, \ldots, \phi^{\rm p},\partial_a\phi^0,\partial_a \phi^1, \ldots,\partial_a \phi^{\rm p}\right).
\eea
The derivatives are with respect to worldsheet coordinates $\overline{x}^a$, with $a = 0, \ldots, {\rm p}$. The background has a metric $g_{\mu\nu}$ with $\mu = 0, \ldots, {\rm p}+q$. Note that the worldsheet has co-dimension $q$. The background induces a metric on the worldsheet 
\bea
\label{eq:induces-met-hyper}
\overline{g}_{ab} = g_{\mu\nu} {x^{\mu}}_{,a}{x^{\nu}}_{,b}.
\eea
The determinant $\overline{g}$ of the worldsheet metric is used as the measure with which to integrate the Lagrangian density to give the action:
\bea
S = \int \dd^{{\rm p}+1}\overline{x}\sqrt{|\overline{g}|}\,\ld.
\eea
The worldsheet metric must have Lorentzian signature, and the field gradients must be independant. This means that the fields can be used as a set of coordinates on the worldsheet, 
\bea
\label{hyoer-coord-phi}
\overline{x}^a = \phi^a,
\eea
and thus the Lagrangian density $\ld$ will depend on just the undifferentiated values $\phi^a$ and on the set of induced metric components $\overline{g}_{ab}$ (there will be $\tfrac{1}{2}{\rm p}({\rm p}+1)$ of these).

The worldsheet energy-momentum tensor will always be given by
\bea
\label{hyper-EMT}
T^{ab} = \frac{2}{\sqrt{-\overline{g}}} \pd{\sqrt{-\overline{g}}\ld}{\overline{g}_{ab}},
\eea
and the corresponding hyper-elasticity tensor on the world-sheet is
\bea
\label{hyper-ET}
\mathfrak{C}^{abcd} = \frac{1}{\sqrt{-\overline{g}}}\pd{\sqrt{-\overline{g}}T^{ab}}{\overline{g}_{cd}}.
\eea
We should note that the hyper-elasticity tensor is related to the elasticity tensor $E^{abcd}$, but it is not exactly the same. There is a  minor, and one major difference. First, there is a multiplicative factor of ``$-2$'' that distinguishes them, $\mathfrak{C}^{abcd}  \sim - 2E^{abcd}$. Secondly, the elasticity tensor is purely orthogonal (i.e., purely spatial in the sense that $u_aE^{abcd}=0$), whereas the hyper-elasticity tensor is not.
These worldsheet tensors can be pulled-back to give  background space-time tensors via
\bse
\bea
T^{\mu\nu} = T^{ab} {x^{\mu}}_{,a}{x^{\nu}}_{,b} = 2\pd{\ld}{\overline{g}_{\mu\nu}} + \ld \overline{g}^{\mu\nu},
\eea
\bea
\mathfrak{C}^{\mu\nu\alpha\beta} = \mathfrak{C}^{abcd}  {x^{\mu}}_{,a}{x^{\nu}}_{,b}{x^{\alpha}}_{,c}{x^{\beta}}_{,d} = \pd{T^{\mu\nu}}{\overline{g}_{\alpha\beta}} + \tfrac{1}{2}T^{\mu\nu}\overline{g}^{\alpha\beta}.
\eea
\ese
The first fundamental tensor of the worldsheet, $\overline{g}^{\mu\nu}$, is constructed from the induced metric $\overline{g}^{ab}$ via
\bea
\overline{g}^{\mu\nu}= \overline{g}^{ab}{x^{\mu}}_{,a}{x^{\nu}}_{,b}.
\eea
When the codimension $q=0$, $\overline{g}^{\mu\nu} = g^{\mu\nu}$, and $\overline{g}{^{\mu}}_{\nu} = {\delta^{\mu}}_{\nu}$. In general, $\overline{g}{^{\mu}}_{\nu}$ is a projector, giving rise to a worldsheet derivative operator
\bea
\overline{\nabla}_{\nu} = \overline{g}{^{\mu}}_{\nu}\nabla_{\mu}.
\eea

In addition to the energy-momentum, first fundamental, and hyper-elasticity tensors, there is another tensor which plays an important role in governing the dynamics: the hyper-Hadamard tensor, given by
\bea
\mathfrak{H}^{\mu\nu\rho\sigma} = g^{\mu\rho}T^{\nu\sigma} + 2 \mathfrak{C}^{\mu\nu\rho\sigma}.
\eea
The standard decomposition of the metric into tangential and orthogonal parts
\bea
g_{\mu\nu} = \overline{g}_{\mu\nu} + \perp_{\mu\nu}
\eea
implies an associated decomposition of the hyper-Hadamard tensor into worldsheet tangential and orthogonal parts,
\bea
\mathfrak{H}^{\mu\nu\rho\sigma} = \overline{\mathfrak{H}}{}^{\mu\nu\rho\sigma} + {^{\perp}\mathfrak{H}}^{\mu\nu\rho\sigma} .
\eea
The worldsheet tangential part is
\bea
\overline{\mathfrak{H}}{}^{\mu\nu\rho\sigma} =  \overline{\mathfrak{H}}{}^{abcd}{x^{\mu}}_{,a}{x^{\nu}}_{,b}{x^{\rho}}_{,c}{x^{\sigma}}_{,d} ,
\eea
with the components of the worldsheet hyper-Hadamard tensor given by
\bea
\label{eq:wthht}
\overline{\mathfrak{H}}{}^{abcd} = \overline{g}^{ac}T^{bd} + 2 \mathfrak{C}^{abcd}.
\eea
The worldsheet orthogonal part is
\bea
{^{\perp}\mathfrak{H}}^{\mu\nu\rho\sigma}  = \perp^{\mu\rho}T^{\nu\sigma}.
\eea

The full space-time metric decomposes first into a piece which is confined to the world sheet, and one which is perpendicular; the confined piece then decomposes into the spatial part and time-like part in the usual fashion:
\bea
g_{\mu\nu} &=& \overline{g}_{\mu\nu} + \perp_{\mu\nu},\\
&=& \gamma_{\mu\nu} - u_{\mu}u_{\nu} + \perp_{\mu\nu}.
\eea

Now that we have established a technology for dealing with these hyper-elastic materials, we can move back to the main goal of describing what we actually mean by hyper-elastic. For the system to ``be'' classified as hyper-elastic, the system should include a sub-system of ordinary elastic type. That is, all scalar fields except one ($\phi^0$, say) only have space-like gradients; i.e., their configuration gradients satisfy the orthogonality condition (\ref{eq:sec:ortho-condition}). What this means is that
\bse
\bea
\label{eq:sec:single-out-phi0}
u^a{\phi^A}_{,a}=0,\qquad \mbox{with}\qquad A= 1,\ldots, {\rm p},
\eea
with $\overline{g}_{ab}u^au^b=-1$, but
\bea
u^a{\phi^0}_{,a}\neq 0.
\eea
\ese
We will use this notation: the small letters ``$a,b,\ldots$'' denote arbitrary worldsheet coordinates, but the larger letters ``$A,B,\ldots$'' denote the coordinate system specified in terms of (\ref{hyoer-coord-phi}) by
\bea
\label{eq:worldsheet-special-1}
\overline{x}^0 = \phi^0,\qquad \overline{x}^A = \phi^A.
\eea
Using (\ref{eq:worldsheet-special-1}) it is apparent that there are  three ``types'' of components of the induced metric (\ref{eq:induces-met-hyper}):
\bse
\label{eq:sec:things-for-lag-to-depend-on-hyper}
\bea
\label{eq:sec:things-for-lag-to-depend-on-hyper-a}
\overline{g}^{00} = \overline{g}^{ab}\mu_a\mu_b,
\eea
\bea
\overline{g}^{A0} = \overline{g}^{ab}{\psi^A}_{a}\mu_b,
\eea
\bea
\overline{g}^{AB} = \overline{g}^{ab}{\psi^A}_{a}{\psi^B}_{b}.
\eea
\ese
we set
\bse
\bea
\label{def-mu-cov}
\mu_a = {\phi^0}_{,a},
\eea
\bea
{\psi^A}_a = {\phi^A}_{,a}.
\eea
\ese
These are the components of the induced metric on which the Lagrangian $\ld$ can depend. The Lagrangian $\ld$ will be a function of
\bea
\ld = \ld\left(\overline{g}^{00},\overline{g}^{A0},\overline{g}^{AB}\right).
\eea
Note that (\ref{eq:sec:single-out-phi0}) implies that
\bea
u^a {\psi^A}_{a} = 0.
\eea
One can perhaps recognise $\mu_a$ as defined in (\ref{def-mu-cov}) as a space-time covector whose components are the space-time derivatives of a scalar, $\mu_a = \partial_a\phi^0$. We will explicitly refer to this interpretation later on, but also note that $\overline{g}^{00}$ as defined in (\ref{eq:sec:things-for-lag-to-depend-on-hyper-a}) corresponds to the ``kinetic term'' which would appear in the canonical action for the scalar field $\phi^0$; $\overline{g}^{00} = \mu^a\mu_a = \partial^a\phi^0\partial_a\phi^0$.


Since $\overline{g}^{AB}$ are the components of a ${\rm p}\times {\rm p}$ matrix there are ${\rm p}$ scalar invariants that can be formed from $\overline{g}^{AB}$ alone. There is one invariant that can be formed from $\overline{g}^{A0}$ alone, and $\overline{g}^{00}$ is already an invariant. This is just something to bear in mind before we explicitly give the catalogue of invariants.

By (\ref{eq:sec:single-out-phi0}), it follows that
\bea
\label{eq:sec:overlineg-gamma}
\overline{g}^{AB} = \gamma^{ab}{\psi^A}_{a}{\psi^B}_{b}= \gamma^{AB},
\eea
after writing the worldsheet metric in the usual way,
\bea
\overline{g}^{ab} = \gamma^{ab} - u^au^b.
\eea


Since the Lagrangian only depends on the field gradients ${\phi^a}_{,b}$ via the induced metric components, the generic variation in the Lagrangian density (\ref{eq:sec:lag-gen-hyper-nsjk}) will be given by
\bea
\label{eq"lag-var-x-g}
\delta\ld = \pd{\ld}{\phi^a}\delta\phi^a+L_{ab} \delta\overline{g}^{ab}.
\eea
in which we defined the partial derivatives of $\ld$ with respect to $\overline{g}^{ab}$ as
\bse
\bea
\label{L_ab_under-one}
L_{ab} = \pd{\ld}{\overline{g}^{ab}} .
\eea
The contravariant components of $L^{ab}$ are given by
\bea
 L^{ab}=-\pd{\ld}{\overline{g}_{ab}} .
\eea
\ese
The tensor $L_{ab}$ is a worldsheet tensor.
After separating out the coordinates (\ref{eq:worldsheet-special-1}), the variation (\ref{eq"lag-var-x-g}) becomes
\bea
\label{var-lag-hyper}
\delta\ld = \pd{\ld}{\phi^0}\delta\phi^0 + \pd{\ld}{\phi^A}\delta\phi^A + L_{00}\delta\overline{g}^{00} +2L_{A0}\delta\overline{g}^{A0} +L_{AB}\delta\overline{g}^{AB}.
\eea



These are the required ingredients for computing the energy-momentum tensor (\ref{hyper-EMT}). One finds
\bea
\label{eq:sec:etm-hyper-general}
T^{ab} = -2L^{ab} + \overline{g}^{ab}\ld.
\eea
One should note that the energy-momentum tensor (\ref{eq:sec:etm-hyper-general}) is not of the form of a perfect solid which we gave in (\ref{eq:sec:emt-solid}). Writing the energy-momentum tensor in terms of the energy density $\rho$, heat flux $q^a$, and pressure tensor $p^{ab}$ without loss of generality,
\bea
T^{ab} = \rho u^au^b + 2q^{(a}u^{b)} + p^{ab}
\eea
one obtains
\bse
\bea
\label{eq:sec:hyper-en-hfdkhfg}
\rho = - \left( \ld + 2 u_au_bL^{ab}\right),
\eea
\bea
\label{eq:sec:hyer-heat}
q^a = {\gamma^a}_bu_cL^{bc},
\eea
\bea
p^{ab} = \gamma^{ab}\ld - 2 {\gamma^a}_c{\gamma^b}_dL^{cd}.
\eea
\ese
The existence of the $L_{A0}$ term in (\ref{var-lag-hyper})   gives rise to the heat flux contribution (\ref{eq:sec:hyer-heat}). Later on we will give an example where the Lagrangian density separates into  ``scalar'' and ``solid'' terms with no interaction between the two: this will switch off the heat flux. The $L_{00}$ term also gives rise to the extra contribution to the energy density (\ref{eq:sec:hyper-en-hfdkhfg}) compared to the equivalent expression for a perfect solid which we gave in (\ref{CQPF-rho}). 

We can use this to compute the class of forces to the conservation equations due to the explicit mixing term, $L_{A0}$. Suppose that we write the energy-momentum tensor (\ref{eq:sec:etm-hyper-general}) as
\bea
T^{ab} = \breve{T}^{ab} + \widehat{T}^{ab},
\eea
in which the first term includes the pure time-like and pure space-like contributions present from the solid and scalar without any mixing, and the second term is the term explicitly present due to their mixing:
\bse
\bea
\label{mizing-Tab-breve}
\breve{T}^{ab} = \rho u^au^b  + p^{ab},
\eea
\bea
\label{mizing-Tab-hat}
\widehat{T}^{ab} = 2q^{(a}u^{b)}.
\eea
\ese
One of the important points we want to emphasise about this construction is that $\widehat{T}^{ab}$ contains the deviations of the full energy-momentum tensor $T^{ab}$ from its perfect solid form (note that the perfect solid, fluid, and scalar field cases are contained within an expression of the form $\breve{T}^{ab}$).
Then, the conservation of $T^{ab}$ implies that the divergence of (\ref{mizing-Tab-breve}) satisfies a forced conservation equation given by
\bea
\nabla_a\breve{T}^{ab} = f^b,
\eea
in which we defined the force to be proportional to the divergence of the mixing contribution to the overall energy-momentum tensor,
\bea
f^b \defn -\nabla_a\widehat{T}^{ab}.
\eea
By computing the divergence of (\ref{mizing-Tab-hat}) one finds
\bea
-f^b = u^b \overline{\nabla}_aq^a + {\gamma^b}_a \dot{q}^a + 2{K^c}_a{\gamma^{(a}}_cq^{b)},
\eea
from which one obtains the two independent projections of the force: 
\bse
\bea
\label{force-projs-a}
u_af^a = \overline{\nabla}_aq^a,
\eea
\bea
\label{force-projs-b}
{\gamma^c}_b f^b = - {\gamma^c}_a\dot{q}^a - 2 {K^d}_a{\gamma^{(c}}_dq^{a)}.
\eea
\ese
Notice that (\ref{force-projs-a}) informs us that $q^a$ is required to be space-varying for the time-like projection of the conservation of $\breve{T}^{ab}$ to be broken. We remind that $q^a$ is defined from the $L_{ab}$ via (\ref{eq:sec:hyer-heat}).

To be able to compute the hyper-elasticity tensor (\ref{hyper-ET}) one also needs
\bea
L_{abcd} = \pd{L_{ab}}{\overline{g}^{cd}} = \pd{^2\ld}{\overline{g}^{ab}\partial\overline{g}^{cd}},
\eea
which gives
\bea
\pd{^2\ld}{\overline{g}_{ab}\partial\overline{g}_{cd}} = L^{abcd} +L^{a(c}\overline{g}^{d)b}+\overline{g}^{a(c}L^{d)b}.
\eea
Using these ingredients, the hyper-elasticity tensor  (\ref{hyper-ET}) is given by
\bea
\label{hyper-elast-gen}
\mathfrak{C}^{abcd} &=& 2\left(L^{abcd} +L^{a(c}\overline{g}^{d)b}+\overline{g}^{a(c}L^{d)b}\right) \nonumber\\
&&\qquad- \left( L^{ab}\overline{g}^{cd} + \overline{g}^{ab}L^{cd} \right) + \tfrac{1}{2}\ld \left(\overline{g}^{ab}\overline{g}^{cd} - 2 \overline{g}^{a(c}\overline{g}^{d)b} \right). 
\eea
Using (\ref{hyper-elast-gen}) the hyper-Hadamard tensor (\ref{eq:wthht}) is given by
\bea
\label{eq:hyper-hadamar-1}
\overline{\mathfrak{H}}{}^{abcd} = 4L^{abcd} + 2 L^{ac}\overline{g}^{bd} + 2 L^{a[d}\overline{g}^{b]c} + 2 \overline{g}^{a[d}L^{b]c} + L\overline{g}^{a[d}\overline{g}^{b]c}.
\eea

When we computed the generic variation of the Lagrangian density (\ref{eq"lag-var-x-g}) we already used the fact that Lagrangian density only depends on the field gradients via the induced metric components. Reversing this gives the Eulerian variation of the Lagrangian, which is the variation with respect to fixed values of the background coordinates $x^{\mu}$ and metric $g_{\mu\nu}$. The variation is
\bea
\ep \ld = \pd{\ld}{\phi^a}\delta\phi^a + {J^a}_b \delta{\phi^b}_{,a},
\eea
where the currents are given in terms of the $L_{ab}$, defined in (\ref{L_ab_under-one}), via
\bea
\label{eq:sec:currents-hype}
{J^a}_b = 2 \overline{g}^{ac} L_{bd}{\phi^b}_{,c}.
\eea
The variational field equations are therefore
\bea
\overline{\nabla}_b{J^b}_a = \pd{\ld}{\phi^a}.
\eea
If we again separate out the coordinates as in (\ref{eq:worldsheet-special-1}) then the currents (\ref{eq:sec:currents-hype}) read
\bse
\bea
{J^a}_0 = 2 \overline{g}^{ab}\left(L_{00}{\phi^0}_{,b} + L_{A0}{\phi^A}_{,b}\right),
\eea
\bea
{J^a}_A = 2 \overline{g}^{ab}\left(L_{A0}{\phi^0}_{,b} + L_{AB}{\phi^B}_{,b}\right).
\eea
\ese

\subsubsection{The symplectic current, and evaluation of sound speeds}
Let $\delta x^{\mu} = \xi^{\mu}$ specify the background coordinate displacements, for whom conservation of the energy-momentum tensor is a consequence. Now suppose that $\delta x^{\mu} = \eta^{\mu}$ is another solution, for example that which results from a symmetry. Then the equation of motion of the generic perturbation vector $\xi^{\mu}$ is given by an expression
\bea
\overline{\nabla}_{\mu}\Omega^{\mu}=0.
\eea
The symplectic current $\Omega^{\mu}$ is given by
\bea
\Omega^{\mu}\left\{\xi,\eta \right\} = \eta^{\mu}{\mathfrak{O}^{\nu}}_{\mu}\left\{\eta\right\} - \xi^{\mu}{\mathfrak{O}^{\nu}}_{\mu}\left\{\eta\right\}
\eea
in which the ``hyper-Hadamard operator'' is defined via
\bea
{\mathfrak{O}^{\nu}}_{\mu}\left\{\xi\right\} = {\mathfrak{H}_{\mu}}^{\nu}{}_{\rho}{}^{\sigma}\overline{\nabla}_{\sigma}\xi^{\rho},
\eea
and the hyper-Hadamard tensor is given in terms of the hyper-elasticity tensor via
\bea
 {\mathfrak{H}_{\mu}}^{\nu}{}_{\rho}{}^{\sigma}=  g_{\mu\rho}T^{\nu\sigma} + 2  {\mathfrak{C}_{\mu}}^{\nu}{}_{\rho}{}^{\sigma}
\eea


Consider the worldsheet hypersurface which has normal vector
\bea
\lambda_a = \lambda_{\mu}{x^{\mu}}_{,a},\qquad \lambda^{\mu}{\perp^{\mu}}_{\nu}=0.
\eea
Then the characteristic equation governing the second derivative of the perturbation vector $\xi^{\mu}$ is given by
\bea
\left[\overline{\nabla}_{\mu}\overline{\nabla}_{\nu}\xi^{\rho} \right]^+_- = \lambda_{\mu}\lambda_{\nu}\zeta^{\rho},
\eea
in which $\zeta^{\mu}$ is a measure of the discontinuity  whose speed we are about to measure: these are the speeds of the wavefronts. The discontinuity of the divergence of the symplectic current is
\bea
\left[\overline{\nabla}_{\mu}\Omega^{\mu} \left\{\xi,\eta\right\}\right]^+_- = \eta^{\mu} {\mathfrak{H}_{\mu}}^{\nu}{}_{\rho}{}^{\sigma}\lambda_{\nu}\lambda_{\sigma}\zeta^{\rho}.
\eea
When $\overline{g}_{\mu\nu}\zeta^{\nu}=0$, i.e., $\zeta^{\mu}$ is worldsheet orthogonal, then the characteristic vector $\lambda_{\mu}$ must be a null eigenvector of $T^{\mu\nu}$:
\bea
\lambda_{\mu}\lambda_{\nu}T^{\mu\nu}=0.
\eea
When $\zeta^{\mu}$ is tangential to the worldsheet, the characteristic equation is expressible in terms of worldsheet tensors via
\bea
\zeta^{\mu} = \zeta^a{x^{\mu}}_{,a},
\eea
with
\bea
\mathcal{Q}_{ab}\zeta^b=0
\eea
and where $\mathcal{Q}_{ab}$ is called the characteristic matrix, defined via
\bea
\mathcal{Q}_{ac} = \overline{\mathfrak{H}}{_a}^b{}_c{}^d\lambda_b\lambda_d.
\eea
The condition for $\zeta^b$ to be an intrinsic characteristic vector is therefore that $\det(\mathcal{Q}_{ab})=0$.
Using (\ref{eq:hyper-hadamar-1}) for the hyper-Hadamard tensor, the characteristic matrix is given by
\bea
\label{characterisic-matrix-defn}
\mathcal{Q}_{ac} = 2 \left( L_{ac}\overline{g}^{bd} + 2 {L_a}^b{}_c{}^d\right) \lambda_b\lambda_d.
\eea
\subsubsection{Separable case}
A simplified example is where the Lagrangian splits as
\bea
\label{eq:sec:lag-sep-proto}
\ld(\phi^0,\ldots, \phi^p,{\phi^0}_{,a},\ldots,{\phi^p}_{,a})  = \qsubrm{\ld}{s}(\phi^0,{\phi^0}_{,a}) + \qsubrm{\ld}{e}(\phi^1, \ldots, {\phi^p},{\phi^1}_{,a},\ldots,{\phi^p}_{,a}).
\eea
That is,  $\qsubrm{\ld}{s}$ is only dependent on $\phi^0$ and its space-time gradient (which includes its time-like gradient), and $\qsubrm{\ld}{e}$ is only dependent of the scalars which don't have a time-like gradient. This restruction  makes $\qsubrm{\ld}{e}$ the Lagrangian density for an elastic solid. With this setup, $\qsubrm{\ld}{s}$ is   the Lagrangian for a ``normal'' scalar field (generically of $k$-essence type), since it depends only on $\phi^0$ and
\bea
\mu_a = {\phi^0}_{,a},
\eea
the gradient 1-form. 
Hence, (\ref{eq:sec:things-for-lag-to-depend-on-hyper-a}) becomes
\bea
\overline{g}^{00} = \overline{g}^{ab}\mu_a\mu_b = - \mu^2.
\eea
This is just the ``kinetic scalar'' corresponding to the scalar field $\phi^0$.
Thus, in a more familiar language, we are asking for the Lagrangian density (\ref{eq:sec:lag-sep-proto}) to split as
\bea
\ld = \qsubrm{\ld}{s}\left( \phi^0, - \mu^2\right) + \qsubrm{\ld}{e}\left(\phi^A, \overline{g}^{AB}\right).
\eea
The first term is the general Lagrangian for a $k$-essence scalar field theory, and the second is a general elastic solid Lagrangian of the type we have discussed throughout the entire review.

Since we are working in the separable case, neither $\qsubrm{\ld}{s}$ or $\qsubrm{\ld}{e}$ will depend on $\overline{g}^{0A}$; this means that the ``entrainment'' effect vanishes, and 
\bea
L_{A0}=0
\eea
 in (\ref{var-lag-hyper}). For the other terms in the variation of the Lagrangian (\ref{var-lag-hyper}) one   obtains
\bse
\bea
L_{00} = \pd{\ld}{\overline{g}^{00}} = - \pd{\qsubrm{\ld}{s}}{\mu^2},
\eea
and
\bea
\label{eq:sec:sep-press-find}
L_{AB} = \pd{ {\ld}{ }}{\overline{g}^{AB}} = \tfrac{1}{2}\left( \qsubrm{\ld}{e} \gamma_{AB} - P_{AB}\right),
\eea
\ese
where $P_{AB}$ is the pressure tensor of the medium, definable as
\bea
\label{sep-press-case-dhsjkhk1}
P^{AB} = \frac{2}{\sqrt{|\gamma|}}\pd{\sqrt{|\gamma|}\qsubrm{\ld}{e}}{\gamma_{AB}},
\eea
in clear analogy with the worldsheet energy-momentum tensor (\ref{hyper-EMT}). Note that we have used (\ref{eq:sec:overlineg-gamma}) to set $\overline{g}_{AB} = \gamma_{AB}$.
One then obtains the elasticity tensor
\bea
{E^{AB}}_{CD} = 2\pd{P^{AB}}{\overline{g}^{CD}} - P^{AB}\gamma_{CD}.
\eea
Hence, since $P_{ab} = P_{AB}{\phi^{A}}_{,a}{\phi^B}_{,b}$, one obtains from (\ref{eq:sec:sep-press-find}) the contravariant components of the pressure tensor on the worldsheet in arbitrary coordinates,
\bea
\label{hyoer-sep-P}
P^{ab} = \qsubrm{\ld}{e}\gamma^{ab} - 2 \gamma^{ac}\gamma^{bd}L_{cd},
\eea
and the elasticity tensor
\bea
E^{abcd} = \left( \gamma^{ab}\gamma^{cd} - 2 \gamma^{a(c}\gamma^{d)b}\right)\qsubrm{\ld}{e} + P^{a(c}\gamma^{d)b} + \gamma^{a(c}P^{d)b} - 4 \gamma^{ae}\gamma^{bf}\gamma^{cg}\gamma^{dh}L_{efgh}.
\eea

The total worldsheet energy-momentum tensor can   be written in separated form as
\bea
\label{eq:sec:sep-emt}
T^{ab} = \qsubrm{T}{s}^{ab} + \qsubrm{T}{e}^{ab},
\eea
in which  the scalar field contribution is
\bse
\bea
\label{eq:sec:sep-emt-scal}
\qsubrm{T}{s}^{ab}  = \qsubrm{\ld}{s}\overline{g}^{ab} - 2 \qsubrm{L}{s}^{ab},
\eea
and the contribution due to the elastic medium is
\bea
\qsubrm{T}{e}^{ab} =\qsubrm{\rho}{e}u^au^b + P^{ab},
\eea
\ese
with $\qsubrm{\rho}{e} =  - \qsubrm{\ld}{e}$, $P^{ab}$ as given by (\ref{hyoer-sep-P}) and
\bea
\qsubrm{L}{s}^{ab} = -\qsubrm{L}{s}'\mu^a\mu^b,\qquad \qsubrm{L}{s}' = - L_{00}.
\eea
The energy density $\qsubrm{\rho}{s}$ and isotropic pressure $\qsubrm{P}{s}$ of the scalar contribution can be read off from (\ref{eq:sec:sep-emt-scal}) as
\bea
\qsubrm{\rho}{s} = 2 \mu^2\qsubrm{L}{s}' - \qsubrm{\ld}{s},\qquad \qsubrm{P}{s} = \qsubrm{\ld}{s}.
\eea

In analogue with the separated energy-momentum tensor (\ref{eq:sec:sep-emt}), the hyper-elasticity tensor can also be expressed in separated form,
\bea
\mathfrak{C}^{abcd} = \qsubrm{\mathfrak{C}}{s}^{abcd}+\qsubrm{\mathfrak{C}}{e}^{abcd}.
\eea
The scalar and elastic contributions are
\bse
\bea
\qsubrm{\mathfrak{C}}{s}^{abcd} &=&  2\left(\qsubrm{L}{s}^{abcd} +\qsubrm{L}{s}^{a(c}\overline{g}^{d)b}+\overline{g}^{a(c}\qsubrm{L}{s}^{d)b}\right) \nonumber\\
&&\qquad- \left( \qsubrm{L}{s}^{ab}\overline{g}^{cd} + \overline{g}^{ab}\qsubrm{L}{s}^{cd} \right) + \tfrac{1}{2}\qsubrm{\ld}{s} \left(\overline{g}^{ab}\overline{g}^{cd} - 2 \overline{g}^{a(c}\overline{g}^{d)b} \right),
\eea
\bea
\qsubrm{\mathfrak{C}}{e}^{abcd} &=& - \tfrac{1}{2}E^{abcd} + 2u^{(a}P^{b)(c}u^{d)} - \tfrac{1}{2}\left( P^{ab}u^cu^d + P^{cd}u^au^b\right) \nonumber\\
&&\qquad+ \tfrac{1}{2}\qsubrm{\rho}{e} \left( \overline{g}^{ab}\overline{g}^{cd} - 2 \overline{g}^{a(c}\overline{g}^{d)b}\right),
\eea
\ese
in which
\bea
\qsubrm{L}{s}^{abcd} = \qsubrm{L}{s}''\mu^a\mu^b\mu^c\mu^d,\qquad \qsubrm{L}{s}'' = \pd{\qsubrm{L}{s}'}{\mu^2}.
\eea



In this separated case the cross-component of the characteristic matrix (\ref{characterisic-matrix-defn}) vanishes
\bea
\mathcal{Q}_{0A}=0.
\eea
This gives a decoupling of the characteric modes $\zeta^a$ for the scalar and elastic parts.
\subsubsection{Reduction to the fluid case}
We would like to remark about the reduction of the theory to include only perfect fluids. A perfect fluid is characterised by the Lagrangian only being a function of the determinant of the material metric: in the language of this section, that is 
\bea
\qsubrm{\ld}{e} = \qsubrm{\ld}{e}(|\gamma_{AB}|),
\eea
where $|\gamma_{AB}|$ denotes the determinant of $\gamma_{AB}$. It is useful to recall the identity
\bea
\pd{|\gamma|}{\gamma^{AB}} = - |\gamma|\gamma_{AB}.
\eea
In this case the spatial parts of (\ref{L_ab_under-one}) evaluate to
\bea
L_{AB} = - \qsubrm{L}{F}\gamma_{AB},\qquad \qsubrm{L}{F} \defn |\gamma|\pd{\qsubrm{\ld}{e}}{|\gamma|},
\eea
and the equivalent of (\ref{sep-press-case-dhsjkhk1}) is
\bea
P_{AB} = \left( \qsubrm{\ld}{e} + 2 \qsubrm{L}{F}\right)\gamma_{AB}.
\eea
The pressure tensor becomes
\bea
P^{ab} = \qsubrm{P}{F} \gamma^{ab},
\eea
with the pressure scalar being given by
\bea
\qsubrm{P}{F} = \qsubrm{\ld}{e} + 2 \qsubrm{L}{F}.
\eea

We would like to offer a generalization, and consider the Lagrangian density
\bea
\ld = \ld\left( \overline{g}^{00},\overline{g}^{0A},|\overline{g}^{AB}|\right).
\eea
We remember that we can still write $\overline{g}^{00} = - \mu^2$.
One then obtains for the components (\ref{L_ab_under-one})
\bse
\bea
L_{00} &=&- \pd{\ld}{\mu^2},\\
L_{0A}&=& \pd{\ld}{\overline{g}^{0A}},\\
L_{AB} &=&-\qsubrm{L}{F}\overline{g}_{AB},
\eea
\ese
where we set
\bea
\mu^2 = - \overline{g}^{00},\qquad \qsubrm{L}{F}\defn  |\overline{g}^{CD}|\pd{\ld}{|\overline{g}^{CD}|}.
\eea
\subsubsection{Non-separable case}
In the general case, $\ld$ will be a function of all invariants formed out of the components (\ref{eq:sec:things-for-lag-to-depend-on-hyper}) of the induced metric (\ref{eq:induces-met-hyper}). And so, regarding $\overline{g}^{ab}$ as a rank-2 tensor in $p+1$ dimensions, there are $p+1$ invariants that can be constructed. The first few such invariants are
\bse
\bea
I_0 &=& \det \rbm{\overline{g}},\\
I_1 &=& \left[\rbm{\overline{g}}\right],\\
I_2 &=&\frac{1}{2!}\left( \left[\rbm{\overline{g}}\right]^2 -  \left[\rbm{\overline{g}}^2\right]\right),\\
I_3 &=&\frac{1}{3!}\left( \left[\rbm{\overline{g}}\right]^3 -  3\left[\rbm{\overline{g}}^2\right]\left[\rbm{\overline{g}}\right]+2\left[\rbm{\overline{g}}^3\right]\right),\\
I_4 &=& \frac{1}{4!}\left( \left[\rbm{\overline{g}}\right]^4+8\left[\rbm{\overline{g}}\right]\left[\rbm{\overline{g}}^3\right] +3\left[\rbm{\overline{g}}^2\right]^2-6\left[\rbm{\overline{g}}\right]^2\left[\rbm{\overline{g}}^2\right]-6\left[\rbm{\overline{g}}^4\right]\right).
\eea
\ese
These are elementary symmetric polynomials.
The Cayley-Hamilton theorem imposes 
\bea
I_{p+1}=I_0, 
\eea
and sets all higher $I_n$ to zero. Note that
\bse
\bea
I_1 =  \overline{g}{^0}_0 + \overline{g}{^A}_A,
\eea
\bea
I_2  =   \overline{g}{}{^0}_0\overline{g}{}{^A}_A-      \overline{g}{}{^A}_0\overline{g}{}{^0}_A +\half \left[ \overline{g}{}{^A}_A\overline{g}{}{^B}_B -  \overline{g}{}{^A}_B\overline{g}{}{^B}_A  \right]
\eea
\ese


\subsection{Generalization away from orthogonal mappings}
Our aim here is to use the previous sections results and ideas to construct a theory for whom the orthogonality of the mappings (\ref{eq:sec:ortho-condition}) doesn't hold. We will keep to the notion of a material metric, $k_{AB}$, and coordinates $\phi^A$ on the material space; we carry on using (\ref{eq:sec:config_gradient}) to define the configuration gradients:
\bea
{\phi^A}_{,a} = {\psi^A}_a.
\eea
The pull-back of the material metric to space-time is
\bea
k_{ab} = k_{AB}{\phi^A}_{,a}{\phi^B}_{,b}.
\eea
The energy-momentum tensor is
\bea
T_{ab} = \ld g_{ab} - 2L_{ab},
\eea
where
\bea
L_{ab} \defn \pd{\ld}{g^{ab}}.
\eea
It follows that $L_{ab}$ can be thought of as the pull-back of some material-manifold tensor $L_{AB}$
\bea
L_{ab} = L_{AB}{\phi^A}_{,a}{\phi^B}_{,b},\qquad L_{AB} \defn \pd{\ld}{k^{AB}}.
\eea
Hence, the energy-momentum tensor is
\bea
T_{ab} = \ld g_{ab} - 2 L_{AB}{\phi^A}_{,a}{\phi^B}_{,b}.
\eea

Suppose we wanted to compute $u^au^bL_{ab}$. Then
\bea
u^au^bL_{ab} = L_{AB}u^a{\phi^A}_{,a}u^b{\phi^B}_{,b},
\eea
but
\bea
u^A = u^a{\phi^A}_{,a}
\eea
is the push-forward of $u^a$. In coordinate free notation the push-forward takes contravariant tensors in space-time and returns a contravariant tensor on the material space via a schematic expression of the form $B^{AB\cdots} = \psi_{\star} B^{ab\cdots}$, as we explained in the discussion leading up to (\ref{eq:sec:pusg-frwad-explanation}).
Hence
\bea
u^au^bL_{ab} =u^Au^B L_{AB}.
\eea
Similarly, if we errected a vector $l^a$ in space-time that is orthonormal to $u^a$, i.e., $l^au_a=0$ and $l^al_a=1$, then
\bea
l^aL_{ab} = l^AL_{AB}{\phi^B}_{,b}
\eea
where
\bea
l^A = l^a{\phi^A}_{,a}
\eea
is the push-forward of $l^a$. We will also introduece a tensor $\gamma^{AB}$ on the material manifold, to be the tensor for whom $l^A$ is an eigenvector,
\bea
l^A{\gamma^{B}}_A = l^B.
\eea
We can now use this to obtain a complete decomposition of the allowed freedom in $L_{AB}$:
\bea
L_{AB} = \left(u_Cu_DL^{CD}\right)u_Au_B + 2 \left( u_Cl_DL^{CD}\right)u_{(A}l_{B)} + \gamma_{A(C}\gamma_{D)B}L^{CD}.
\eea


In the hyper-elastic category one has the splitting
\bea
\left({\phi^A}_{,a}\right) \longrightarrow \left( {\phi^0}_{,a},{\phi^{\overline{A}}}_{,a}\right)
\eea
in which
\bea
u^a{\phi^0}_{,a} \neq 0,\qquad u^a {\phi^{\overline{A}}}_{,a} = 0.
\eea
This means that the pull-back of the material metric  splits up as
\bea
k_{ab} = k_{00}{\phi^0}_{,a}{\phi^0}_{,b} + 2 k_{\overline{A}0}{\phi^0}_{,a}{\phi^{\overline{A}}}_{,b} +k_{\overline{A}\overline{B}}{\phi^{\overline{A}}}_{,a} {\phi^{\overline{B}}}_{,b} .
\eea
There are three projections of the pulled-back material metric:
\bse
\bea
u^au^bk_{ab} = u^au^bk_{00}{\phi^0}_{,a}{\phi^0}_{,b} ,
\eea
\bea
u^a{\gamma^b}_ck_{ab} = 2u^a{\gamma^b}_c k_{\overline{A}0}{\phi^0}_{,a}{\phi^{\overline{A}}}_{,b} ,
\eea
\bea
{\gamma^a}_c{\gamma^b}_d k_{ab}={\gamma^a}_c{\gamma^d}_b k_{\overline{A}\overline{B}}{\phi^{\overline{A}}}_{,a} {\phi^{\overline{B}}}_{,b}.
\eea
\ese
It is convenient, and does not lose generality, to set
\bea
\mu_a \defn {\phi^0}_{,a},\qquad l_a\defn k_{\overline{A}0}{\phi^{\overline{A}}}_{,a},\qquad \overline{k}_{ab} \defn  k_{\overline{A}\overline{B}}{\phi^{\overline{A}}}_{,a} {\phi^{\overline{B}}}_{,b},
\eea
in which
\bea
u^al_a=0,\qquad u^a\overline{k}_{ab} =0.
\eea
The pull-back of the material metric now naturally splits up as
\bea
k_{ab} = \mu_a\mu_b + 2 \mu_{(a}l_{b)} + \overline{k}_{ab}.
\eea

Suppose we wrote $\overline{k}_{ab}$ as the pull-back with respect to some new map $\widehat{\psi}$ of some new tensor, $\overline{k}_{ab}=\widehat{\psi}^{\star}\overline{k}_{AB}$, where in coordinates
\bea
\overline{k}_{ab} = \overline{k}_{\widehat{A}\widehat{B}}{\widehat{\phi}^{\widehat{A}}}{}_{,a}{\widehat{\phi}^{\widehat{B}}}{}_{,b},
\eea
and the components of this new configuration gradient satisfy
\bea
u^a{\widehat{\phi}^{\widehat{A}}}{}_{,a}=0.
\eea

\cleardoublepage
\section{Final discussion}
\begin{itemize}
\item Whats the point?
\item what have we learnt?
\item where next?
\end{itemize}